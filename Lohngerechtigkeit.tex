\documentclass[a4paper,12pt]{article}
\usepackage{fullpage}
\renewcommand{\baselinestretch}{1.1}

% Sprache
\usepackage[ngerman]{babel}
%\usepackage[babel,german=quotes]{csquotes}

% Links
\usepackage[hidelinks]{hyperref}
\urlstyle{rm}

% Bibliographie
\usepackage[longnamesfirst]{natbib}
\bibpunct[:]{(}{)}{;}{a}{}{,}
\setlength{\bibsep}{0.0pt}
%\usepackage[style=authoryear-comp,natbib=true,url=false,isbn=false,doi=false]{biblatex}
%\addbibresource{bib.bib}

% Tabellen
\usepackage{tabularx,booktabs,dcolumn,ragged2e}
\usepackage[figuresright]{rotating}
\let\belowcaptionskip=\abovecaptionskip
\newcolumntype{L}[1]{>{\RaggedRight\arraybackslash\hspace{0pt}}p{#1}}
\newcolumntype{C}[1]{>{\centering\arraybackslash\hspace{0pt}}p{#1}}
\newcolumntype{R}[1]{>{\RaggedLeft\arraybackslash\hspace{0pt}}p{#1}}
\newcolumntype{x}{>{\RaggedRight\arraybackslash\hspace{0pt}}X}
\newcolumntype{z}{>{\Centering\arraybackslash\hspace{0pt}}X}
\newcolumntype{H}[1]{>{\RaggedRight\arraybackslash\hangindent1em\hspace{0pt}}p{#1}}
\newcommand{\armultirow}[3]{%
    \multicolumn{#1}{#2}{\begin{picture}(0,0)\put(0,0){%
    \begin{tabular}[t]{@{}#2@{}}#3\end{tabular}}\end{picture}}}
\makeatletter\let\expandableinput\@@input\makeatother % https://tex.stackexchange.com/questions/144625/misplaced-noalign-error-in-table-but-only-when-using-include

% Graphiken
\usepackage{graphicx}
%\usepackage{tikz}
%\usetikzlibrary{arrows}

% Kommentare
\usepackage{xcolor}
\newif\ifcomments
\commentsfalse
\commentstrue % auskommentieren, um die Kommentare zu unterdrücken
\newcommand{\comment}[1]{%
    \ifcomments\marginpar{\renewcommand{\baselinestretch}{1}\tiny\hspace*{-1.1em}\colorbox{gray!20}%
    {\textcolor{red}{\parbox[t]{.9in}{\raggedright #1}}}}\fi}
\newcommand{\alert}[1]{\ifcomments\textcolor{red}{#1}\else#1\fi}


% Encoding/Schrift
\usepackage[utf8]{inputenc}
\usepackage[T1]{fontenc}

% Satz
\pagestyle{plain}
\clubpenalty=10000
\widowpenalty=10000
\displaywidowpenalty=10000
\sloppy

% Inhalt
\begin{document}

% Titlepage
\title{Gerechte Löhne für Frauen und Männer. Ergebnisse von drei Experimenten}
\author{Ben Jann, Barbara Zimmermann und Andreas Diekmann}
\date{\today}
\maketitle

% Inhaltsverzeichnis
\tableofcontents

\newpage
\begin{abstract}
    \comment{Deutschen Abstract hinzufügen ...}
    
    %Just wages for men and women. Findings from three Experiments (Abstract)
    The gender pay gap in Switzerland slightly diminished during the last two
    decades, but the gap is still substantial. In 2014, women earned about 18\%
    less than men in a full time job in the private sector. \comment{Zahlen aktualisieren?}
    About 40\% of this gap cannot be explained
    by differences in relevant factors such as human capital or economic
    sectors and it remains unclear whether the unexplained gender wage gap is
    due to discrimination or whether the applied statistical models are simply
    not accurate enough. To shed light on the gender wage gap from a different
    viewpoint we used an alternative approach and conducted three factorial
    survey experiments to determine whether there is a “just wage gap” between
    men and women. The results from the first experiment provide evidence for a
    gender-based double standard in assessing incomes: Test persons judged
    lower earnings for women as just. That is, according to these judgments,
    women should indeed earn less than men. The findings from the second
    experiment stand in contrast these results: Employing a similar design, no
    gender differences in just earnings could be found. A reason for the
    disagreement might be that the specified household context was not held
    constant across experiments. We therefore conducted a third experiment in
    which we systematically varied whether the described persons were single or
    married. The results indicate that the family context is indeed an
    important mediator. No gender gap could be found for singles, but higher
    earnings were judged as fair for married men than for married women. That
    is, according to the opinions of the respondents, 
    there seems to be a “marriage premium” for men.
\end{abstract}

\newpage
% Einleitung
\section{Einleitung}
\label{sec:s1}

\comment{Hinweis auf Heiratsrecht bis 1988: “Note that this arrangement was enshrined 
in Switzerland’s marriage law until 1988, giving husbands the status as legal 
head of family from whom wives needed the consent in order to take on paid employment.” 
(Paper von Benita und Oli Oesch im ESR; das paper sollte irgendwo zitiert werden)}

\comment{cite „Oesch:2017“ somewhere; see bibtex file}

\comment{OK, beide Artikel zitiert.}

Gemäss dem letzten Bericht des Bundesamts für Statistik verdienten Frauen im
Jahr 2016 im Schweizer Privatsektor im Durchschnitt 19.6\% weniger als Männer
(gemessen an auf ein 100\%-Pensum standardisierten Löhnen) \comment{Zahlen aktualisieren? -> Zahlen aktualisiert.}
\citep{BFS-2019a}. Zum Vergleich: Als das Bundesamt für Statistik die
Lohnungleichheit 1998 zum ersten Mal anhand der Lohnstrukturerhebung (LSE)
untersuchen liess, betrug die Differenz fast 25\% \citep{Strub-etal-2006}. Der
Unterschied hat sich also über die Jahre verkleinert, ist jedoch immer noch
beachtlich (für eine Analyse im Zeitverlauf siehe z.B. \citealp{Schmid-2016}).

Wieso hält sich der geschlechtsspezifische Lohnunterschied so hartnäckig?
Ansätze zur Erklärung des Gender-Pay-Gap sind vielfältig und kommen aus der
Ökonomie, Soziologie und vermehrt auch aus der Sozialpsychologie (für eine
Übersicht siehe \citealp{Blau-Kahn-2017} oder
\citealp{Weichselbaumer-Winter-Ebmer-2005}). Nachfolgend fassen wir die
wichtigsten Argumente und Ergebnisse zusammen.
\comment{Die nachfolgenden Ausführungen stark verdichten; max 2-3 Seiten}

%\subsection{Lohnungleichheit aus ökonomischer und soziologischer Perspektive}

Die vermutlich längste Tradition zur Erklärung von Lohnungleichheiten haben die
ökonomischen Theorien, insbesondere die Humankapitaltheorie \citep{Becker-1975,
Mincer-1958,Mincer-Polachek-1974}. Das Humankapital bezeichnet die Akkumulation
von Fähigkeiten, Erfahrungen und Bildungsabschlüssen, die ein Individuum
besitzt und die sich -- zumindest gemäss Theorie -- positiv auf seine
Produktivität auswirken. Demnach werden Lohnungleichheiten damit erklärt, dass
sich Menschen in ihrem Humankapital und folglich in ihrer Produktivität
unterscheiden. 
Insgesamt hat sich das Bildungsniveau zwischen Frauen und Männern über die letzten Jahrzehnte angeglichen. 28\% der 25-64-jährigen Frauen haben einen Hochschulabschluss. Bei den Männern sind es knapp 30\%. Bei den 25-34-jährigen haben mit 42\% inzwischen deutlich mehr Frauen als Männer (35\%) einen Hochschulabschluss \citep{BFS-2019b}. 

Frauen arbeiten häufiger Teilzeit und haben aufgrund von familiären Verpflichtungen mehr Erwerbsunterbrechungen, was ihre Möglichkeiten, Arbeitserfahrung zu sammeln, beeinträchtigt. Dies resultiert in einer reduzierten Humankapitalausstattung und somit in Lohneinbussen. 

Die Theorie geht noch einen Schritt weiter: Da Frauen eine eher
diskontinuierliche Erwerbslaufbahn antizipieren, treffen sie entsprechende
Bildungsentscheidungen und wählen zum Beispiel Berufe, in denen die
Vereinbarkeit von Arbeit und Familie einfacher möglich ist oder in denen
Erwerbsunterbrüche mit einer geringen Entwertung des Humankapitals einhergehen
\citep{Polachek-1981}.

\comment{diesen Abschnitt in den Theorieteil?}
Diese ungleiche Verteilung der Geschlechter auf Berufe und Tätigkeiten -- horizontale Segregation genannt -- ist in der Schweiz sehr ausgeprägt und hat sich über die Zeit nur wenig verändert \citep{Buchmann-Kriesi-2012,Charles-2005,Schwiter-etal-2014}.
Gemäss der sogenannten „Devaluations-Hypothese“
\citep{England-etal-1988,Liebeskind-2004} wird von Frauen ausgeübte Arbeit
gesellschaftlich weniger hoch bewertet als typisch männlich konnotierte Arbeit. Die
Abwertung erfolgt zum Beispiel dadurch, dass Qualifikationen, welche für die
häufig von Frauen ausgeübten Berufe erforderlich sind, nicht also solche
wahrgenommen, sondern als inhärent angesehen werden. Folglich schlagen sich
diese Qualifikationen auch nicht in einem höheren Lohn nieder
\citep{England-1992,England-2005,Gottschall-1995}. Die geschlechtsspezifische
Prägung eines Berufes kann sich über die Zeit ändern, und tatsächlich gibt es
Forschung, die einen negativen Zusammenhang zwischen der Feminisierung
(verstanden als Zunahme des Frauenanteils) eines Berufs und dessen
durchschnittlichem Einkommen findet (z.B.
\citealp{Levanon-etal-2009,Murphy-Oesch-2016}).

Die bessere Ausbildung und die Zunahme der Erwerbstätigkeit haben einen Beitrag
zur Verminderung der Lohnungleichheit zwischen Frauen und Männern geleistet
(siehe auch \citealp{Jann-Engelhardt-2008}). Der Umstand, dass Frauen weiterhin
öfter in tieferen Pensen Teilzeit arbeiten und für einen Grossteil der
unbezahlten Haus- und Familienarbeit zuständig sind, verhindert jedoch ihre
gleichwertige Arbeitsmarktteilnahme. Während die geschlechtsspezifische
Rolleinteilung besonders in der neoklassischen Ökonomie als rationale
Kosten-Nutzen-Überlegung dargestellt wird („Spezialisierungs-Hypothese“,
\citealp{Becker-1981}), kritisieren SoziologInnen dies als strukturelles
Problem. Die „doppelte Vergesellschaftung“ \citep{Becker-Schmidt-2008}, ein
Produkt der Industrialisierung, bezeichnet eine doppelte Einbindung der Frauen
in das Sozialgefüge, einerseits auf dem Arbeitsmarkt und andererseits im
Haushalt. Die daraus resultierende doppelte Belastung ist nicht zum Nutzen der
Frauen. Daneben, dass der Haus- und Familienarbeit wenig Wert zugesprochen
wird, verhindert sie auch die gleichwertige Integration in den Arbeitsmarkt.

\cite{Oesch:2017} zeigen anhand von Paneldaten, dass Frauen starke Lohneinbussen erfahren, sobald sie Kinder haben, wovon ein grosser Teil mit der Reduktion der Arbeitszeit erklärt werden kann. Allerdings bleibt ein Anteil von bis zu 9\% des Lohnunterschieds zwischen Müttern und kinderlosen Frauen unerklärt. Mit einem zusätzlichen Vignetten-Experiment bei Personalverantwortlichen, zeigen die Autoren, dass Müttern im Vergleich zu ansonsten vergleichbaren Frauen ohne Kinder auch ein tieferer Lohn zugesprochen wird und diese seltener angestellt würden. 

Die tieferen Frauenlöhne lassen sich allerdings nicht allein mit diesem sogenannten „motherhood penalty“  \citep{Budig-2001} erklären. In ihrer neuen Studie, finden \cite{Combet-2019} Lohnunterschiede zwischen Frauen und Männern lange bevor diese eine Familie gründen und beginnen, sich die Hausarbeit ungleich aufzuteilen. Dabei bleibt ein unerklärter Unterschied vorhanden, auch wenn eine Vielfalt von möglichen Einflussfaktoren, darunter Humankapital, Job-Charakterisitka und Werte in Bezug auf Arbeit und Familie kontrolliert werden. 

Ein Teil des Lohnunterschieds erklärt sich weiterhin dadurch, dass Frauen
seltener in Führungspositionen gelangen als Männer. In der Schweiz betrug der
Frauenanteil unter den ArbeitnehmerInnen mit einer Führungsfunktion oder in
einer Unternehmensleitung im Jahr 2018 36\% \citep{BFS-2019c}. In den
Geschäftsleitungen der 100 grössten Schweizer Firmen befanden sich im selben
Jahr nur 7\% Frauen. In Verwaltungsräten betrug ihr Anteil immerhin 19\%
\citep{Schillingreport-2018}. Dabei ist unklar, ob es einfach eine Frage der
Zeit ist, bis sich die noch jungen Bildungserfolge der Frauen in entsprechenden
Positionen auf dem Arbeitsmarkt niederschlagen, oder ob weiterhin unsichtbare
Barrieren vorhanden sind, die Frauen am Aufstieg hindern (eine so genannte
„gläserne Decke“). Forschende, die diesen Ansatz verfolgen, erklären das
Vorhandensein der gläsernen Decke unter anderem mit dem Einfluss sozialer
Netzwerke, von denen Männer mehr profitieren (für die Schweiz siehe z.B.
\citealp{Rost-2010}), oder mit Homophilie, bzw. homosozialer Reproduktion, also
der Präferenz von ManagerInnen (wobei es eher Manager sind), ihresgleichen zu
fördern (siehe z.B. \citealp{Bihagen-Ohls-2006,Holst-Wiemer-2010,Ochsenfeld-2012}). Frauen erzielen
also tiefere Löhne, weil sie seltener Führungspositionen innehaben.
Gleichzeitig ist allerdings der Lohnunterschied zwischen den Geschlechtern auf
den oberen Hierarchiestufen grösser als auf den tieferen Stufen
(\citealp{Blau-Kahn-2016}, \citealp[42]{Strub-Bannwart-2017}).

\section{Bisherige Literatur und theoretische Argumente}
\label{sec:s2}

Trotz umfassenden statistischen Modellen, welche die oben ausgeführten Erklärungsfaktoren berücksichtigen, verbleibt ein unerklärter Lohnunterschied zwischen Frauen und Männern. Dieser betrug im Zeitraum zwischen 1998 und 2016 im Durchschnitt ca. 40\% des Gesamtunterschieds \citep{BFS-2019a,Strub-2010}. Ein Trend in die eine oder andere Richtung ist nicht auszumachen. Dabei wird kontrovers debattiert ob dieser unerklärte Anteil nun durch Diskriminierung oder wegen ungenau spezifizierten statistischen Modellen zustande kommt.

\subsection{Diskriminierung}
\comment{Abschnitt wurde neu überarbeitet}

Theoretische Überlegungen sprechen durchaus dafür, dass Diskriminierungen zumindest nicht auszuschliessen sind. Die meistbeachteten theoretischen Ansätze basieren auf Gary
Beckers „Economics of Discrimination“ \citep{Becker-1973}. Demnach erfolgt Diskriminierung aufgrund von Vorurteilen gegenüber bestimmten Gruppen („taste-based discrimination“). Das heisst, bei gleicher Produktivität wird eine bestimmte Gruppe, beispielsweise Frauen, seltener befördert oder sie erhalten einen tieferen Lohn. Diese diskriminierende Haltung kann von drei unterschiedlichen Akteuren ausgehen: Den ArbeitgeberInnen, den anderen MitarbeiterInnen oder den KundInnen des Unternehmens. In einem perfekt kompetitiven Markt sollte auf Abneigung basierte Diskriminierung nicht vorkommen, da diskriminierende ArbeitgeberInnen aus dem Markt gedrängt würden. Nach \cite{Arrow-1972} sagt Beckers Modell folglich genau die Abweseheit des Phänomens voraus, das es zu erklären sucht (siehe auch  \citealp{Guryan-2013}).

Das Konzept der „statistischen Diskriminierung“ \citep{Arrow-1972,Phelps-1972} versucht
die unterschiedliche Behandlung von Frauen und Männern dahingehend zu erklären,
dass sich Akteure wegen mangelnder Informationen bei der Festlegung des Lohns
an statistischen Gruppenmerkmalen orientieren. Das bedeutet, je weniger Informationen über die Produktivität eines bestimmten Individuums vorhanden sind, desto mehr Gewicht erhalten Gruppenmerkmale. Bei perfekter Information müsste die Diskriminierung also verschwinden. Diese Theorie kann
Lohnungleichheiten zwischen den Geschlechtern im Prinzip nur dann erklären, wenn
sich die Produktivität der beiden Geschlechter effektiv unterscheidet. Soziologische und
sozialpsychologische Studien zeigen allerdings, dass diese These nicht haltbar
ist (z.B. \citealp{Bielby-Bielby-1988}). Im Gegenteil, auch wenn Frauen in
Job-Evaluationen bessere Bewertungen erhalten, sind es die Männer, welche dann
trotzdem eher befördert werden \citep{Blau-DeVaro-2007} oder grössere
Lohnerhöhungen erhalten \citep{Castilla-2012}. Interessant ist das Konzept der 
statistischen Diskrimierung jedoch, weil die herangezogenen Gruppenbewertungen auch auf 
Stereotypen beruhen können, die nicht der Wahrheit 
entsprechen, und sich solche Stereotype aufgrund selbstverstärkender Prozesse 
verfestigen (und unter Umständen im Sinne einer selbsterfüllenden Prophezeiung 
sogar bewahrheiten) können.

In diese Richtung geht auch die „status construction theory“, manchmal auch „rewards expectation theory“ \citep{Auspurg-etal-2017} genannt, 
\citep{Ridgeway-1997,Ridgeway-2001} \comment{kann man das gleichsetzen?}. Gemäss dieser haben bereits existierende Status- und Ressourcenunterschiede zwischen verschiedenen sozialen Gruppen (insb. zwischen
Frauen und Männern) Auswirkungen auf die Interaktionen zwischen den Individuen
dieser Gruppen. Dabei entstehen sogenannte „status beliefs“, also Glaubenssätze
über den Status der Mitglieder der jeweiligen Gruppen, welche in Interaktionen
laufend reproduziert werden (vgl. \citealp{Goffman-1977}). Diese Glaubenssätze
über den sozialen Status von Frauen und Männern haben zur Folge, dass Frauen
weniger kompetent eingeschätzt werden und ihnen ein tieferer Status
zugesprochen wird, was ungleiche Behandlung zum Beispiel auf dem Arbeitsmarkt
zur Folge hat \citep{Ridgeway-1997}. Es ist also eine Art ein Teufelskreis:
Weil Männer öfter in höheren Positionen sind, wird ihnen ein gesellschaftlich
höherer Status zugeschrieben, was es wiederum als legitim erscheinen lässt,
dass sie höhere Positionen oder höhere Einkommen erhalten (siehe dazu auch
\citealp{Berger-etal-1972}). 

\comment{Abschnitt zu Segregation \& Devaluation hier (siehe comment unten)?}

\subsection{Verteilungsgerechtigkeit} \comment{diesen Abschnitt auch kürzen (und schärfen)}

\comment{statistische D. hat viel mit Vorstellungen zu tun; ähnliche Hinweise
bei Literatur zur Abwertung von Frauenberufen; oder auch die Theorien zur
Spezialisierung aufgrund von Rollen}

Diese in den vorangehenden Abschnitten beschriebenen Theorien und empirischen
Studien, versuchen zu erklären, warum Lohnunterschiede zwischen Frauen und
Männern existieren, obwohl diese in einer offenen und auf meritokratischen
Prinzipien basierenden Gesellschaft, gar nicht existieren sollten und gemäss
Gesetz auch nicht existieren dürften. Diesen Widerspruch zwischen den
bestehenden und bisher nur teilweise erklärten Lohnunterschieden und dem
Anspruch einer modernen Gesellschaft auf Chancengleichheit werden wir nun etwas
genauer betrachten. Dabei rücken die Prinzipien der Verteilungsgerechtigkeit in
den Fokus. Wann wird die Verteilung von knappen Ressourcen, bspw. von
Einkommen, als gerecht beurteilt und welche Kriterien kommen dabei zur
Anwendung? Ist es gerecht, dass ein persönliches Merkmal, wie das Geschlecht
einen Einfluss darauf hat, wie viel eine Person von einem bestimmten Gut
erhält? Oder sollen alle gleich viel erhalten?

Bereits in der Philosophie der Antike wurden Fragen zur
Verteilungsgerechtigkeit diskutiert. Dabei haben sich mehrere, sich teilweise
widersprechende Kriterien etabliert. Die drei bekanntesten Kriterien sind
Gleichheit, Verdienst und Bedürftigkeit (vgl. z.B.
\citealp{Deutsch-1975,Miller-1992,Sabbagh-2001}). Gleichheit bedeutet „für alle
gleich viel“, ohne irgendwelche Charakteristiken der Personen zu
berücksichtigen. Die Wichtigkeit des Kriteriums basiert u.a. auf den Werken von
David Hume und kommt heute vor allem im Recht zur Anwendung (gleiche Rechte für
alle). Für die Einkommensgerechtigkeit spielt es zumindest in der
kapitalistischen Marktwirtschaft eine untergeordnete Rolle.

Dass „allen das Gleiche“ nicht immer gerecht ist, erkannte bereits Aristoteles.
Sein in der „nikomachischen Ethik“ eingeführtes Gerechtigkeitsprinzip basiert
auf einer „proportionalen Gleichheit“. Für ihn muss „eine gewisse Würdigkeit
das Richtmass der distributiven Gerechtigkeit sein“
\citep[107]{Aristoteles-1985}. Worin diese Würdigkeit nun besteht, lässt er
weitgehend offen. In der philosophischen Literatur wird dieses Prinzip nun
meist „Verdienst-Prinzip“ genannt und kann verschiedene Aspekte von Verdienst
oder Leistung enthalten, zum Beispiel Anstrengung, Fähigkeiten oder Erfolg.
Welche dieser Kriterien wieviel zählen sollen, darüber herrscht keine Einigkeit
\citep{Lamont-1994}.

Das dritte Prinzip ist das der „Bedürftigkeit“. Demnach sollen diejenigen mehr
von etwas bekommen, die einen höheren Bedarf danach haben. Dieses Prinzip hat
seine Bedeutung heute hauptsächlich im Zusammenhang mit sozialstaatlichen
Massnahmen. So gibt es Leistungen, wie beispielsweise die Sozialhilfe oder die
Ergänzungsleistungen, welche an die Bedürftigkeit der EmpfängerInnen geknüpft
sind. Die Leistungen der Altersvorsorge hingegen (insbesondere die 2. Säule)
basieren eher auf dem Verdienst-Prinzip und werden (unter gewissen
ausgleichenden Mechanismen, v.a. in der 1. Säule) gemäss den durch
Erwerbsarbeit erwirtschafteten Vorsorgeguthaben verteilt.

Dass askriptive Merkmale wie Geschlecht, soziale Herkunft, ethnische
Zugehörigkeit oder sexuelle Orientierung in zeitgenössischen Theorien zur
Verteilungsgerechtigkeit nicht vorkommen und bei der Beurteilung eines
gerechten Einkommens keine Rolle spielen sollten, ist kein Zufall.
\comment{(ggfs. Begründung mit Rawls, „Schleier des Nichtwissens“ oder so;
Rawls 1979)} Demzufolge können Lohnunterschiede zwischen Frauen und
Männern erst einmal nicht gerecht sein, ausser es werden Gründe gefunden,
wonach Frauen entweder weniger verdienen oder weniger benötigen als Männer. In
Übereinstimmung mit der Humankaptialtheorie
\citep{Becker-1975,Mincer-Polachek-1974} soll eine höhere Produktivität zu
einem höheren Einkommen führen. Damit also Frauen gerechterweise weniger
verdienen als Männer, müsste ihre Produktivität tiefer sein. Dies wurde in
einer Vielzahl empirischer Studien untersucht und konnte nicht belegt werden.
Bei gleichen Voraussetzungen weisen Frauen die gleiche Produktivität auf wie
Männer.

Auch in Bezug auf das zweite Gerechtigkeitskriterium, Bedürftigkeit, leuchtet
es nicht auf Anhieb ein, wieso Frauen weniger verdienen sollten. Oder führen
sie etwa einen bescheideneren Lebensstil als Männer? Das häufigste Argument
diesbezüglich ist, dass verheiratete Frauen auf das Einkommen ihres Mannes
zählen können und deshalb weniger darauf angewiesen sind, ein gleichwertiges
Einkommen zu generieren. Diese Überlegungen basieren auf der
geschlechtsspezifischen Rollenteilung, welche dem Mann die Erwerbsarbeit
zuweist und die Frau für die Hausarbeit und Kindererziehung vorsieht. Auch wenn
sich in den letzten Jahrzehnten diesbezüglich viel verändert hat, die
Grundfesten wurden nicht erschüttert. So waren im Jahr 2018 82,5\% der erwerbstätigen Männer in einem Vollzeit Pensum (90-100\%) beschäftigt, hingegen traf dies lediglich auf 41.4\% der erwerbstätigen Frauen zu \citep{BFS-2019d}.
Dementsprechend leisten die Frauen auch einen weitaus grösseren Teil der unbezahlten Haus- und Familienarbeit. So leisten 15-64-jährige Frauen im Mittel knapp 30 Studen Haus- und Familienarbeit pro Woche. Bei den Männern sind es 18 Stunden \citep{BFS-2017}. \comment{Hier evtl. etwas ergänzen.-> Habe ein paar Zahlen und Erläuterungen ergänzt}
Dass Frauen weniger verdienen, wenn sie weniger bezahlte Erwerbsarbeit leisten, mag einleuchten. Dass sie aber für gleichwertige Arbeit (in Inhalt und Umfang) weniger Lohn erhalten, lässt sich vorerst nur schlecht damit erklären, dass sie eben mehr Zeit für die Familie aufwenden als Männer. Auch die umgekehrte These, dass verheiratete Männer bei der Entlöhnung einen sogenannten „Heiratsbonus“ erhalten, wird kontrovers diskutiert. \comment{die folgenden beiden Abschnitte sind aus dem Kapitel von Exp. 3 ausgeschnitten und verschoben worden.} Diverse Studien zeigen, dass verheiratete Männer im Durchschnitt mehr verdienen als alleinstehende
\citep{Budig-Lim-2016,Killewald-Gough-2013}. Gemäss Beckers
(\citeyear{Becker-1981}) Spezialisierungs-Hypothese liegt der Grund darin, dass
es ökonomisch sinnvoll sei, wenn sich verheiratete Männer auf die Erwerbsarbeit
konzentrieren und sich Frauen aufgrund des komparativen Vorteils bezüglich des
Kinderkriegens auf die Haus- und Familienarbeit spezialisieren. Diese These ist
allerdings sehr umstritten. Die Untersuchungen von \citet{Ludwig-Bruederl-2011}
und \citet{Jakobsson-Kotsadam-2016} zeigen etwa, dass ein Selektionseffekt
vorliegt, nach dem Männer mit höherem Einkommenspotenzial öfter heiraten. Die
„Heiratsprämie“ der Männer könnte ein Artefakt dieser Selektion sein.
Alternativ wird vermutet, dass von Seiten der Arbeitgeber"-Innen eine positive
Diskriminierung stattfinden könnte in dem Sinne, dass diese bereit sind,
verheirateten Männern aufgrund von traditionellem Rollendenken (Mann
als Haupternährer) mehr Lohn zu bezahlen (\citealp{Budig-Lim-2016}; vgl. auch
\citealp{Killewald-Gough-2013}).


\subsection{Empirische Gerechtigkeitsforschung}

Um herauszufinden, wie die Leute denken und welche Kriterien ihnen bei
Verteilungsfragen wichtig sind, wurden in der empirischen Sozialforschung in
den letzten Jahren vermehrt Vignetten-Experimente (auch faktorieller Survey
genannt) durchgeführt (eine ausführliche Beschreibung des methodischen Vorgehens bei Vignetten-Experimenten befindet sich im Kapitel \ref{sec:s3}). Eingeführt wurde die Methode aber bereits früher durch
Peter H. Rossi (vgl. \citealp{Rossi-1979,Rossi-Nock-1982}). Aufbauend auf den
Pionierarbeiten im Bereich der Einkommensgerechtigkeit
\citep{Jasso-1980,Jasso-Rossi-1977,Jasso-Webster-1997} wurden verschiedene
Vignetten-Experimente durchgeführt, bspw. um den Einfluss von
Gerechtigkeitsprinzipien zu untersuchen oder das Zustandekommen von
Diskriminierungen besser zu verstehen (z.B. \citealp{Auspurg-etal-2017,Gatskova-2015,Sauer-etal-2009}).

In ihrem Artikel testen \citet{Auspurg-etal-2017} mehrere der in den vorherigen Abschnitten diskutierten Theorien.  Um zu überprüfen, ob statistische Diskriminierung eine Ursache für tiefere Frauenlöhne sein könnten, formulierten sie eines der Vignetten-Experimente so, dass die Befragten Vignetten mit unterschiedlich vielen Informationen über eine fiktive Person erhielten und danach deren Einkommen bewerten mussten. Trifft die Theorie zu, müsste der Lohnunterschied zwischen den bewerteten Frauen- und Männervignetten dann am grössten sein, wenn am wenigsten Information, aus der die Produktivität abgeleitet werden könnte, vorhanden ist. Die Hypothese wurde allerdings nicht bestätigt: Mehr Information führte nicht dazu, dass für Frauen und Männer der gleiche Lohn als gerecht empfunden wurde. Der gender wage gap zu Ungunsten der Frauen blieb vorhanden.
Weiter testen die AutorInnen ob die Theorie zutrifft, dass Frauen und Männern einen unterschiedlichen Status zugeschrieben wird und dies zu einer ungleichen Bewertung der Frauen- und Männervignetten führt. Die „rewards expectation theory“, welche wir oben „status construction theory“ genannt haben, testet genau dies. Wenn es zutrifft, dass Männern mehr Fähigkeiten zugesprochen werden, sollte dies in männerdominierten Berufen besonders der Fall sein, da dort Vorurteile über geschlechtsspezifische Kompetenzen besonders zum tragen kommen. Weiter der Unterschied ebenfalls grösser sein, wenn die Befragten in Berufen mit hoher Lohnungleichheit arbeiten, da existierende Ungleichheiten sich in den Köpfen der Menschen verfestigen und sich somit auch immer wieder reproduzieren \citep[182]{Auspurg-etal-2017}. Diese Hypothesen werden teilweise bestätigt. Es zeigt sich, dass Lohnunterschiede in der Branche der Befragten wie auch in derjenigen der Vignetten-Personen eine Rolle spielen: Je grösser der reale Lohnunterschied ist, desto unterschiedlicher ist auch die Bewertung der Frauen- und Männerlöhne. Hingegen hat der Frauenanteil im Beruf keinen signifikanten Effekt. Diese Erkenntnisse sprechen durchaus dafür, dass „gender status beliefs“ eine Rolle spielen bei der Bewertung von Einkommen. Inwieweit diese aber von einem unterschiedlichen zugeschriebenen Status abhängen oder eher aufgrund der angenommenen geschlechtsspezifischen Rollenteilung zwischen Erwerbs- und Hausarbeit zustandekommen, bleibt weitgehend offen.

Andere experimentelle Forschung belegt, dass der Zivilstand oder die familiäre Situation einen Einfluss auf die Einkommensbewertung hat. Die Untersuchungen von \citet{Gatskova-2015} und 
\citet{Sauer-etal-2009} zeigen, dass das Einkommen von Vignetten-Personen, welche mit
einer nichterwerbstätigen Person verheiratet sind, von den Versuchspersonen
jeweils als zu tief beurteilt wird. Allerdings bleibt in diesen beiden
Experimenten unklar, ob dies für Frauen und Männer gleichermassen gilt. 
\citet{Shamon-Duelmer-2014} diskutieren diesen Aspekt und leiten daraus die Hypothese ab,
dass den Männern aufgrund der geschlechtsspezifischen Arbeitsteilung ein
höheres Einkommen zugestanden wird als den Frauen \citep[348]{Shamon-Duelmer-2014}.
In den Vignetten halten sie den Zivilstand
konstant. Die beschriebenen Frauen und Männer sind alle mit einer
nichterwerbstätigen Person verheiratet. Ihre Ergebnisse zeigen aber gerade das
Gegenteil der postulierten Hypothese: Die Einkommen der Frauen werden als zu
tief bewertet und nicht diejenigen der Männer. \comment{Da könnte auch ein
„Kompensations-Effekt“ eine Rolle spielen, denn dass in verheirateten Paaren
nur die Frau erwerbstätig ist, ist ungewöhnlich und hat vermutlich meist
gewichtige Gründe (z.B. Arbeitslosigkeit oder Krankheit des Mannes).}

\subsection{Unsere Studie}

Wir knüpfen mit unserer Studie an diese Literatur an. Um zu untersuchen, welche
Ursachen der mit statistischen Daten nicht erklärbare Teil des Lohnunterschieds
zwischen Männern und Frauen haben könnte, haben wir im Zeitraum zwischen 2001
und 2010 drei verschiedene und aufeinander aufbauende Vignetten-Experimente
durchgeführt. Diese werden in den folgenden Abschnitten im Detail beschrieben.

\comment{hier z.b. auf Auspurg et al. 2017 aufbauen -> habe ich jetzt weiter oben gemacht...}

Im Experiment 1 untersuchen wir erstens, ob Diskriminierung für
Lohnunterschiede zwischen Frauen und Männern verantwortlich sein könnte.
Zweitens interessiert uns, ob Prinzipien der Verteilungsgerechtigkeit, namentlich Verdienst und Bedürftigkeit, bei der
Beurteilung eines gerechten Lohns zur Anwendung kommen. Dank dem
experimentellen Design können wir Diskriminierung von anderen Ursachen, wie
bspw. unterschiedliches Humankapital isolieren, wir können jedoch nicht genau
sagen, wie diese Diskriminierung zustande kommt. 

In Experiment~2 versuchen wir die Ergebnisse von Experiment~1 in Bezug auf die
Relevanz des Geschlechts zu replizieren. Um weitere mögliche Einflussfaktoren,
die mit der familiären Situation zusammenhängen könnten, zu kontrollieren,
halten wir in diesem Experiment den Zivilstand konstant. Alle
Vignetten-Personen werden als alleinstehend und ohne Kinder beschrieben.
Zweitens interessiert uns, ob die horizontale Segregation einen Einfluss auf
die Einkommensbewertung hat.

In unserem dritten Experiment wollen wir nun überprüfen, ob sich das
Gerechtigkeitsprinzip der Bedürftigkeit im Zivilstand manifestiert und ob dies
gemäss der Spezialisierungs-Hypothese nur für Männer oder auch für Frauen
zutrifft.

\section{Daten und Methode}
\label{sec:s3}

Hier allgemeine sachen: 

- 3 experimente mit faktoriellem survey

- die verschiedenen Datenquellen

\subsection{Experiment 1}
Die Daten stammen aus einer schriftliche Befragung zum Thema „Ungleichheit und
Gerechtigkeit“, die 2001 am Institut für Soziologie der Universität Bern
durchgeführt wurde. Befragt wurden Personen einer Zufallsstichprobe der
deutschschweizerischen Wohnbevölkerung im Alter von mindestens 18
Jahren.\footnote{Es handelte sich um eine einfache Wahrscheinlichkeitsauswahl
von Haushalten aus dem Telefonverzeichnis. Pro Haushalts wurde je eine zu
befragende Zielperson anhand der Geburtstagsmethode bestimmt.} Die
Ausschöpfungsquote beträgt 34\% (531 gültige Interviews). Für detailliertere
Informationen zu der Erhebung und den Daten siehe \citet{Jann-2001}.

Die Befragung enthält eine Reihe von Fragen zu unterschiedlichen Aspekten von
Ungleichheit und Gerechtigkeit. Enthalten ist zudem ein Vignetten-Experiment,
welches darin besteht, eine Vignette -- also eine Beschreibung einer
spezifischen Situation oder Aussage -- zu bewerten. Eine Beispielvignette aus
der Befragung ist in Abbildung~\ref{fig-1} dargestellt. Variiert wurden die
Faktoren \emph{Geschlecht}, \emph{Bedürftigkeit}, und \emph{Leistung}; siehe 
Tabelle~\ref{tab-ex1-faktoren} zu Ausprägungen und Wortlaut der Variationen.
Die anderen in der Vignette angesprochenen Merkmale (Alter, Beruf, Einkommen)
wurden konstant gehalten.

Aus der Variation der drei Faktoren mit je zwei Ausprägungen ergeben sich acht
verschiedene Vignetten-Kombinationen, die den Befragten nach dem Zufallsprinzip
zur Beurteilung zugewiesen wurden. Dadurch wird gewährleistet, dass die
Faktoren der Vignetten nicht mit Eigenschaften der Befragten zusammenhängen und
auch untereinander nicht korreliert sind. Die bivariaten Effekte der
Vignetten-Faktoren auf die Bewertung können also abgesehen von
Zufallsvariationen – und unter dem Vorbehalt, dass anders als in einem
Laborexperiment nicht alle Rahmenbedingungen kontrolliert wurden – als
kausal interpretiert werden (keine Scheinkorrelationen, die Einflüsse
sämtlicher Drittvariablen sind neutralisiert).


\begin{figure}\centering
    \fbox{
    \includegraphics[width=0.8\textwidth]{gr/vignette-2001}
    }
    \caption{Beispiel einer Vignette aus Experiment 1}\label{fig-1}
\end{figure}


\begin{table}
    \small
    \caption{Experimentelle Faktoren in Experiment 1}\label{tab-ex1-faktoren}
    \begin{tabularx}{\textwidth}{@{}llX@{}}
        \toprule
        Faktor          & Ausprägung & Wortlaut       \\\midrule
        Geschlecht      & männlich     & “Herr Meier”   \\
                        & weiblich     & “Frau Meier”   \\
        \addlinespace
        Bedürftigkeit   & tief         & „Er/sie ist verheiratet in kinderloser Ehe. Finanzielle Sorgen kennt er/sie keine.“ \\
                        & hoch         & „Er/sie ist alleinerziehender Vater/alleinerziehende Mutter von zwei Kindern. Finanziell kommt er/sie nur knapp über die Runden.“ \\
        \addlinespace
        Leistung        & tief         & „Sein/ihr berufliches Engagement lässt zu Wünschen übrig und die Anforderungen an seine/ihre Leistung erfüllt er/sie nur knapp“   \\
                        & hoch         & „Er/sie arbeitet engagiert und seine/ihre Aufgaben und Pflichten erfüllt er/sie zur vollen Zufriedenheit seines/ihres Arbeitgebers“  \\
        \bottomrule
    \end{tabularx}
\end{table}


Um Effekten sozialer Erwünschtheit entgegen zu wirken, wurde jeder
Befragungsperson nur eine Vignette zur Bewertung vorgelegt („between“ Design).
Würden mehrere Vignetten mit variierenden Ausprägungen präsentiert, hätten die
Befragten die Möglichkeit, die Alternativen zu vergleichen und die Bewertungen
mit einem sozial erwünschten oder politisch korrekten Antwortverhalten
abzustimmen. Liegt jedoch nur eine Vignette vor, so ist für die befragten
Personen unklar, welche Merkmale variiert werden (bzw. dass überhaupt etwas
variiert wird), und die abgegebenen Bewertungen werden nicht durch
Vergleichsprozesse beeinflusst. Effekte der sozialen Erwünschtheit sollten so
weitgehend ausgeschlossen werden können. 

\subsection{Experiment 2}
Die Datenerhebung fand im Rahmen einer Kurzbefragung zum
Thema „Lohnniveau in der Schweiz“ statt und wurde im Jahr 2006 an der ETH
Zürich durchgeführt. Es handelte sich dabei um eine postalische Befragung einer
Zufallsstichprobe der deutschschweizerischen Wohnbevölkerung ($N = 371$,
Rücklauf: 41.2\%). Die Vignetten bestanden aus einem $2 \times 2 \times 3 \times
3$-Design mit den folgenden Faktoren (eine Beispielvignette findet sich in
Abbildung~\ref{fig-2}): \comment{Tabelle zu Faktoren wie bei Experiment 1}
\begin{itemize}
\item Geschlecht: weiblich vs. männlich
\item Nationalität: schweizerisch klingender Name („Herr/Frau Walter“) vs. ausländisch klingender Name („Herr/Frau Ismailovic“)
\item Beruf: gemischt (Journalist/in), frauendominiert (Krankenpfleger/in), männerdominiert (Schreiner/in)
\item Einkommen: drei Stufen im Abstand von je 500 CHF (Journalist/in: 5000, 5500, 6000; Krankenpfleger/in: 4000, 4500, 5000; 
Schreiner/in: 4500, 5000, 5500)
\end{itemize}

\begin{figure}\centering
    \fbox{
    \includegraphics[width=0.9\textwidth]{gr/vignette-2006}
    }
    \caption{Beispiel einer Vignette aus Experiment 2}\label{fig-2}
\end{figure}

\subsection{Experiment 3}
Experiment~3 wurde 2010 im Rahmen der zweiten Panelbefragung des Umweltsurveys
2007 der ETH Zürich durchgeführt. Befragt wurden Personen einer repräsentativen
Zufallsstichprobe der Schweiz. Von den 2'517 Personen der Bruttostichprobe
(Teilnehmende der Erstbefragung) nahmen 1'945 Personen teil, was einem Rücklauf
von 77.3\% entspricht (für weitere Informationen zur Datenerhebung siehe
\citealp{Diekmann-etal-2012}). Wie in den vorangehenden beiden Befragungen
wurden die Versuchspersonen u.a. gebeten, eine Vignette zu bewerten. Diese wurde
auf den folgenden Dimensionen variiert (siehe Abbildung~\ref{fig-3} für eine Beispielvignette):
\comment{Tabelle zu Faktoren wie bei Experiment 1}
\begin{itemize}
\item Geschlecht: weiblich vs. männlich
\item Zivilstand: alleinstehend ohne Kinder vs. verheiratet ohne Kinder
\item Leistung: durchschnittlich vs. überdurchschnittlich
\item Einkommen: 5000 CHF, 5500 CHF, 6000 CHF
\end{itemize}

\begin{figure}\centering
    \fbox{
    \includegraphics[width=0.95\textwidth]{gr/vignette-2010}
    }
    \caption{Beispiel einer Vignette aus Experiment 3}\label{fig-3}
\end{figure}

- in allen drei designs: jeweils nur eine Vignette (wegen soz. Erwünschth.); Randomisierung der Faktoren (Experiment); (allgemeine Punkte)

Umweltsurvey 2011: Einschränkung auf DS

Übericht zu Fallzahlen etc.

Da wir mit einem experimentellen Design arbeiten und es in erster Linie darum geht,
die kausalen Strukturen aufzudecken, und die Generalisierung i.S. 
repräsentative deskriptive Aussagen über die Population weniger im Vordergrund stehen, verzichten wir auf eine Gewichtung.
Aus dem gleichen Grund verzichten wir auf eine Einschränkung der Stichprobe der dritten Studie auf die Deutschschweiz und 
behalten die Fälle aus der französischen und italienischen Schweiz bei.

Gewichten und Fälle ausschliessen reduziert die Effizienz und somit die Power der Studie...

Die allgemeinen Zusammenhangsstruktur bleibt gleich ..., die Konfindenzintervalle werden aber etwas grösser (weniger Effizienz.)

Die Spezifika der einzelnen Experimente werden bei der Vorstellung der Resultate besprochen...

Teil zu Analysemethode: 
\begin{itemize}
    \item predictive margins from saturated models including all interactions among experiemntal factors; balances random 
fluctuations and thus stabilizes statistical analysis (more power); anaylses based on plain mean differences are in online supplement 
    \item details of Willingness-to-pay analysis (Umrechnung zu Franken)
    \item evtl. Resultate auch noch mit „randomization inference“ replizieren
\end{itemize}


\section{Resultate}

\subsection{Experiment 1}
\comment{Teile der Beschreibung der Daten könnten in den Abschnitt „Daten und Methode“}


Abbildung~\ref{fig-ex1} zeigt die durchschnittlichen Einkommensbewertungen in
Abhängigkeit des Geschlechts der beschriebenen Person, im Total sowie
aufgeschlüsselt nach Bedürftigkeit und Leistung (berichtet sind Regression-Adjustment-Resultate unter 
Berücksichtigung sämtlicher Interaktionseffekte). Alle drei Kriterien --
Geschlecht, Leistung und Bedürftigkeit -- scheinen bei der Beurteilung des
Einkommens eine Rolle zu spielen. Zudem scheint der Effekt des Geschlechts
relativ unabhängig zu sein vom Leistungs- und Bedürftigkeitsniveau.

% Geschlecht

Auf der Skala von --5 („viel zu niedrig“) bis 5 („viel zu hoch“) erhalten
Frauen den Wert --0.61. Ihr durchschnittliches Vignetten-Einkommen wird somit
als etwas zu niedrig eingestuft. Die durchschnittlichen Vignetten-Einkommen der
Männer hingegen werden mit einem Wert von --1.49 deutlicher als zu niedrig
einstuft. Das heisst, wenn es sonst keine Unterschiede gibt zwischen Frauen und
Männern, wird für Frauen ein signifikant tieferes Einkommen als gerecht
empfunden.

\begin{figure}\centering
    \includegraphics[scale=.9]{log/ex1}
    \caption{Einkommensbewertungen nach Geschlecht der beschriebenen Person 
    in Experiment 1 (Mittelwerte inkl. 95\%-Konfidenzintervalle)}\label{fig-ex1}
\end{figure}

% Leistung und Bedürftigkeit

Das Einkommen von Personen, denen in den Vignetten eine hohe Leistung in Form
von Engagement und Arbeitserfüllung zur Zufriedenheit des Arbeitgebers
attestiert wurde, wird häufiger als zu tief eingestuft als dasjenige von
Personen, deren Leistung zu wünschen übrig lässt. Die Mittelwertdifferenz (über
beide Geschlechter) von 1.80 ist gross und signifikant ($p<0.001$). Leistung
soll also belohnt werden. Ein ähnliches Bild zeigt sich bei der Bedürftigkeit.
Diese wurde anhand von zwei unterschiedlichen Kriterien operationalisiert:
Einerseits die familiäre Situation (alleinerziehend oder verheiratet ohne
Kinder) und andererseits die effektiven finanziellen Bedürfnisse („kommt nur
knapp über die Runden“ vs. „kennt keine finanziellen Sorgen“).
Alleinerziehende, die sich in finanziell prekären Verhältnissen befinden sollen
mehr verdienen als verheiratete ohne Kinder, die keine finanziellen Sorgen
haben (Differenz über beide Geschlechter: 1.27, $p<0.001$). Dadurch, dass diese
beiden Kriterien – Kinder und finanzielle Sorgen – nicht auch separat variiert
wurden, kann allerdings nicht festgestellt werden, ob bei der Beurteilung eines
von beiden wichtiger ist, ob beide die gleiche Rolle spielen, oder ob die
Kombination von beiden ausschlaggebend ist. Ein Zusammenhang ist allerdings
anzunehmen, denn es ist bekannt, dass die finanziellen Bedürfnisse von Familien
mit Kindern höher sind als diejenigen von kinderlosen. Zudem sind
Alleinerziehende überdurchschnittlich oft von Armut betroffen und auf
Sozialhilfe angewiesen \citep{Amacker-etal-2015}. Zusammengefasst zeigen unsere
Analysen, dass beide Kriterien der Verteilungsgerechtigkeit, also Leistung und
Bedürftigkeit, bei der Beurteilung eines gerechten Einkommens relevant sind.

% Interaktion Geschlecht & Leistung / Bedürftigkeit

Als nächstes gehen wir der Frage nach, wie die verschiedenen Kriterien der
Verteilungsgerechtigkeit mit dem askriptiven Merkmal Geschlecht in Verbindung
gebracht werden können. Kann der Geschlechterunterschied mit einem der anderen
Kriterien erklärt werden? In den Vignetten wurden Leistung, Bedürftigkeit und
Geschlecht randomisiert. Das heisst, die Leistung und Bedürftigkeit der Frauen
unterscheidet sich nicht systematisch von derjenigen der Männer. Somit ist der
Geschlechterunterschied nicht auf ein unterschiedliches Leistungs- oder
Bedürftigkeitsniveau zurückzuführen. Von Interesse ist jedoch auch die Frage,
ob Leistung und Bedürftigkeit für beide Geschlechter gleich bewertet werden
oder ob das Leistungs- und das Bedürftigkeitsprinzip bei den beiden
Geschlechtern unterschiedlich zur Anwendung kommen. In Abbildung~\ref{fig-ex1}
ist ersichtlich, dass Frauen und Männer bei gleicher Leistung unterschiedlich
bewertet wurden (die Mittelwerte unterscheiden sich signifikant) und dass diese
Differenz bei tiefer Leistung tendenziell etwas grösser ist als bei hoher. Bei
Frauen ist der Wert bei tiefer Leistung positiv, was dahingehend zu
interpretieren ist, dass ihr Einkommen im Schnitt als zu hoch beurteilt wurde.
Bei Männern hingegen ist dies nicht der Fall. Ihr Einkommen wird auch bei
tiefer Leistung als eher zu tief eingestuft. Es gibt also eine leichte Tendenz,
Leistung von Frauen und Männern nicht gleich zu beurteilen. Allerdings sind die
Unterschiede gering und die entsprechenden Differenzen nicht signifikant
($p=0.46$). Wie sieht es bezüglich des Bedürftigkeitsprinzips aus? Auch wenn es
sich bei der Bedürftigkeit um etwas komplett anderes handelt als bei der
Leistung, fallen die Resultate vergleichbar aus. Besteht eine hohe
Bedürftigkeit, wird das Vignetten-Einkommen der Männer signifikant deutlicher
als zu tief bewertet als dasjenige der Frauen, welches ebenfalls als zu niedrig
eingeschätzt wird. Alleinerziehende Männer in knappen finanziellen
Verhältnissen sollen also mehr verdienen als vergleichbare alleinerziehende
Frauen. Bei tiefer Bedürftigkeit, also bei Verheirateten ohne Kinder, wird das
Einkommen der Frauen, im Gegenteil zu dem der Männer, als etwas zu hoch
eingeschätzt (Abbildung~\ref{fig-ex1}). Die Differenz zwischen verheirateten
Frauen und Männern ist etwas grösser als diejenige zwischen Alleinstehenden.
Erklären lässt sich eine solche Differenz mit der in unserer Gesellschaft
weiterhin vorherrschenden geschlechtsspezifischen Rollenteilung, bei der Männer
mehr zum Haushaltseinkommen beitragen und Frauen sich hauptsächlich um den
Haushalt und die Kinder kümmern, woraus ein höherer Einkommensbedarf des Mannes
abgeleitet werden kann. Die Differenz befindet sich jedoch in einer ähnlichen
Grössenordnung wie bei der Leistung und ist ebenfalls nicht signifikant
($p=0.35$).

In Abbildung~\ref{fig-ex1} ist weiterhin der Einfluss des \emph{Geschlecht der
Versuchsperson} dargestellt. Im Total ist der Effekt des Geschlechts der
Versuchsperson ist fast null und nicht signifikant ($p=0.71$), was bedeutet,
dass Frauen und Männer die Vignetten gesamthaft nicht verschieden beurteilten.
Es könnte jedoch sein, dass jeweils das eigene Geschlecht bevorzugt wird.
Deshalb ist in der Abbildung der Effekt des Vignettengeschlechts getrennt nach
Geschlecht der Versuchsperson ausgewiesen. Die männlichen Versuchspersonen
haben tendenziell einen etwas grösseren Unterschied zwischen Frauen und Männern
gemacht als die weiblichen Versuchspersonen, die Differenz ist aber nicht
statistisch signifikant ($p=0.28$). Tatsächlich ist es so, dass auch Frauen
den Männern in den Vignetten ein höheres Einkommen zugestanden haben als den
Frauen. Der Effekt des Geschlechts der Vignette ist für Männer ($p<0.001$) 
wie auch für Frauen ($p<0.007$) signifikant.

Die verschiedenen Analysen haben gezeigt, dass das Geschlecht der Vignette und
die beiden Kriterien der Verteilungsgerechtigkeit -- Verdienst und
Bedürftigkeit -- eine entscheidende Rolle spielen bei der Beurteilung, ob ein
bestimmtes Einkommen als gerecht empfunden wird. Das Hauptresultat von
Experiment~1 ist ohne Zweifel, dass Frauen unter sonst gleichen Umständen
weniger Einkommen zugestanden wird als Männern -- und zwar von beiden
Geschlechtern! Dieses Resultat ist überraschend, würde man in einer
aufgeklärten und nach meritokratischen Prinzipien organisierten Gesellschaft
doch erwarten, dass das Geschlecht bei der Einkommensbewertung keine Rolle
spielt. Entgegen dieser Erwartung scheinen die Befragten die Situationen von
Frauen und Männern jedoch -- zumindest unbewusst -- mit unterschiedlichen
Massstäben bewerten.

Das Experiment liefert allerdings keine Erklärung dafür, warum ein solcher
Unterschied gemacht wird. Weiterhin weist das experimentelle Design eine
Schwäche auf, indem beim Bedürftigkeitskriterium zwei unterschiedliche
Dimensionen vermischt wurden. Es ist deshalb nicht möglich zu beurteilen, ob
die familiäre oder eher die finanzielle Situation für die Versuchspersonen für den 
Effekt des Bedürftigkeit entscheidend war.


\subsection{Experiment 2}
\comment{Teile der Beschreibung der Daten könnten in den Abschnitt „Daten und Methode“}
\comment{Doku der Daten erstellen}

 

\noindent 
Die Resultate des Experiments sind in Abbildung~\ref{fig-ex2}
dargestellt (wiederum Regression-Adjustment-Resultate unter Berücksichtigung
von Interaktionen). Da das Einkommen in der Vignette variiert wurde, lassen sich die 
Ergebnisse auch zu Frankenbeträgen umrechnen; diese Resultate finden sich in 
Abbildung~\ref{fig-ex2chf}. Im Gegensatz zu Experiment~1 spielt das Geschlecht bei der
Bewertung der Einkommen in Experiment~2 keine Rolle. Das in der Vignette
angegebene Einkommen wurde bei Frauen und Männern gleichermassen in
Durchschnitt als etwas zu tief bewertet. Die befragten Personen haben also
keine diskriminierenden Präferenzen gegenüber Frauen geäussert. Auch
aufgeschlüsselt nach Berufen und nach Nationalität lassen sich keine
substanziellen Unterschiede zwischen der Bewertung der weiblichen und der
männlichen Vignette feststellen. Weiterhin hat auch das Geschlecht der
Versuchsperson keinen Einfluss; das heisst, Frauen wie auch Männer haben keinen
Unterschied gemacht zwischen der Bewertung der weiblichen und der männlichen
Vignette.

\begin{figure}\centering
    \includegraphics[scale=.9]{log/ex2}
    \caption{Einkommensbewertungen nach Geschlecht der beschriebenen Person in 
    Experiment 2 (Mittelwerte inkl. 95\%-Konfidenzintervalle)}\label{fig-ex2}
\end{figure}
\begin{figure}\centering
    \includegraphics[scale=.9]{log/ex2chf}
    \caption{Einkommensbewertungen in CHF nach Geschlecht der beschriebenen Person in 
    Experiment 2 (Mittelwerte inkl. 95\%-Konfidenzintervalle)}\label{fig-ex2chf}
\end{figure}


In Experiment~2 finden wir also überraschenderweise keinen geschlechtsspezifischen
Unterschied bei der Einkommensbewertung der Vignetten, obwohl das
Studien-Design sehr ähnlich war wie das von Experiment~1. Wie lässt sich diese
Diskrepanz erklären? In beiden Studie wurde eine repräsentative
Zufallsstichprobe der Deutschschweiz befragt, womit wir ausschliessen können,
dass es sich um Resultate für grundsätzlich verschiedene Populationen handelt.
Auch gehen wir nicht davon aus, dass den unterschiedlichen Ergebnissen
fundamentale Veränderungen in den Einstellungen zu Geschlechtergerechtigkeit
zugrunde liegen; dafür ist der Zeitraum von 2001 bis 2006 zu kurz.

Allerdings könnte der Familienkontext in der Vignette eine Rolle gespielt
haben. In Experiment~2 sind alle Frauen und Männer in den Vignetten
alleinstehend und kinderlos. In Experiment~1 waren sie entweder alleinerziehend
und in einer prekären finanziellen Situation oder kinderlos verheiratet und
ohne finanzielle Sorgen.
\comment{Das Resultat könnte aber auch
    ein Artefakt sein aufgrund der Fragebogenkonstruktion: Bei Exp. 1 sollen die
    Leute in der Frage vor der Vignette angeben, was lohnrelevant sein soll. Zur
    Auswahl stehen Verdienst- und Bedürftigkeitskriterien. Bei Exp. 2 geht es in
    den beiden Fragen vor der Vignette zuerst um die Rechtfertigung von
    Managerlöhnen, im Vgl. zu denen Journi, Schreiner und Krankenschwester
    natürlich alle wenig verdienen und dann wird nach dem angenommenen
    Durchschnittseinkommen gefragt, was die Leute im Mittel auf 8'500.- schätzen,
    also auch viel höher als alle Vignetten. Dass sich deshalb die
    Geschlechterunterschiede aufheben ist jedoch längst nicht zwingend (v.a. weil
    die Bewertung insgesamt nur als leicht und nicht viel zu tief ausfällt).}

Unter Umständen liegt in diesem Designunterschied der Grund für die sich
widersprechenden Ergebnisse. Falls dem so wäre, würde dies gleichzeitig
Hinweise für die Erklärung des in Experiment~1 gefundenen Geschlechtereffekts
liefern. So kann vermutet werden, dass bei der Bewertung des Einkommens die
Befragten berücksichtigen, ob es Hinweise auf eine weitere Person gibt, die zum
Haushaltseinkommen beiträgt. Ein Beitrag einer weiteren Person ist bei
Verheirateten wie auch bei Alleinerziehenden (Alimente) wahrscheinlicher als
bei Alleinstehenden. In Übereinstimmung mit den vorherrschenden
Geschlechterrollen in der Gesellschaft gehen die Befragten von einem grösseren
Beitrag aus, wenn diese weitere Person ein Mann ist. Das heisst, in Experiment~1 
wird implizit eine zusätzliche Einkommenskomponente dazu gedacht, die sich je
nach Geschlecht der Person in der Vignette unterscheidet. Wenn dies zutrifft,
würden wir bei Experiment~1 einen Geschlechtereffekt erwarten, nicht jedoch bei
Experiment~2. Um diese Hypothese zu testen, haben wir ein drittes Experiment
durchgeführt.
\comment{copy-paste aus deinen PPT von Luzern 2014}
\comment{Wobei heutzutage bekommt man kaum mehr Alimente (aber das wissen die
    Leute vielleicht nicht – wusste ich jedenfalls nicht, bis ich da im
    Scheidungsprojekt mit Dorian gearbeitet habe)}


\subsection{Experiment 3}
\comment{Teile der Beschreibung der Daten könnten in den Abschnitt „Daten und Methode“}
\comment{„Offizielle“ Version der Umweltsurveys verwenden}


Die Resultate des Experiments finden sich in Abbildung~\ref{fig-ex3} und –
umgerechnet zu Frankenbeträgen – in Abbildung~\ref{fig-ex3chf}. Die drei in den
Vignetten experimentell variierten Dimensionen -- Geschlecht, Leistung und
Zivilstand -- unterscheiden sich in den Mittelwertsvergleichen jeweils
signifikant. Im Mittel werden alle Einkommen als etwas zu hoch bewertet. Für
Frauen wird, wie in Experiment~1, ein deutlich tieferes Einkommen als gerecht
empfunden als für Männer ($p<0.001$; der Einkommensunterschied entspricht im
Mittel etwa 280 Franken pro Monat). Erwartungsgemäss hat auch die Leistung
einen grossen Einfluss auf die Bewertung des Vignetten-Einkommens. Bei tiefer
Leistung wird das Einkommen deutlicher als zu hoch beurteilt ($p<0.001$; der
Unterschied beträgt etwa 660 Franken). Weiterhin wird das Einkommen von
Alleinstehenden eher als zu hoch beurteilt als dasjenige von Verheirateten
($p=0.008$; im Schnitt ist der Unterschied etwa 170 Franken).

In einem weiteren Schritt interessiert uns, ob der Geschlechtereffekt mit der
Leistung und dem Zivilstand zusammenhängt. Zwischen dem Geschlecht und der
Leistung finden wir -- übereinstimmend mit den Ergebnissen von Experiment~1 --
keinen Interaktionseffekt. Das heisst, bei tiefer wie auch bei hoher Leistung
wird das Einkommen der Männer weniger stark als zu hoch bewertet als das
Einkommen der Frauen ($p=0.005$ bzw. $p=0.001$) bzw. Leistung wird bei beiden
Geschlechtern ähnlich bewertet. Bezüglich des Zivilstands finden wir allerdings
in Einklang mit unserer Hypothese einen signifikanten Interaktionseffekt. Der
Effekt des Zivilstands hat also eine geschlechtsspezifische Komponente:
Verheiratete Männer sollen gemäss den Befragten mehr verdienen als
alleinstehende. Für Frauen gibt es jedoch keine solche „Heiratsprämie“. Anders
formuliert machen die Befragten keinen Unterschied zwischen der männlichen und
der weiblichen Vignette, wenn es sich um Alleinstehende handelt ($p=0.141$).
Geht es jedoch um Verheiratete, wird den Männern ein höheres Einkommen als den
Frauen zugestanden ($p<0.001$). Diese Differenz im Geschlechtereffekt nach
Zivilstand ist signifikant mit einem $p$-Wert von 0.017. Die Resultate passen
gut zur Hypothese, dass es vom Familienkontext abhängt, ob ein Unterschied
zwischen den Geschlechtern gemacht wird. Alleinstehende Frauen ohne Kinder
werden nicht benachteiligt, verheiratete Frauen aber schon. Allerdings scheint
die Vermutung, dass das Einkommen des Partners bei der Bewertung „dazu gedacht“
wird, falsch zu sein, da für Frauen die Einkommensbewertungen unabhängig vom
Zivilstand sind. Vielmehr ist der Unterschied bei den Männern: Verheiratete
Männer sollen gemäss Einschätzung der Befragten etwa 300 Franken mehr Einkommen
erhalten als alleinstehende Männer. Dies steht in Einklang mit dem
traditionellen Rollenbild des Mannes als Haupternährer.

\begin{figure}[p]\centering
    \includegraphics[scale=.9]{log/ex3}
    \caption{Einkommensbewertungen nach Geschlecht der beschriebenen Person in 
    Experiment 3 (Mittelwerte inkl. 95\%-Konfidenzintervalle)}\label{fig-ex3}
\end{figure}

\begin{figure}[p]\centering
    \includegraphics[scale=.9]{log/ex3chf}
    \caption{Einkommensbewertungen in CHF nach Geschlecht der beschriebenen Person in 
    Experiment 3 (Mittelwerte inkl. 95\%-Konfidenzintervalle)}\label{fig-ex3chf}
\end{figure}

\comment{2 Abschnitte zu Heiratsbonus nach oben verschoben.}

Unsere Ergebnisse aus Experiment 3 haben gezeigt, dass Geschlecht und
Zivilstand einen Einfluss darauf haben, wie das Einkommen bewertet wird. Der
Einfluss des Zivilstands ist aber geschlechtsspezifisch: Verheirateten Männern
wird ein höheres Einkommen zugestanden, bei Frauen gibt es keinen Unterschied.
Wir vermuten daher, dass neben der Spezialisierungs-These und dem
Selektionseffekt auch noch eine unterschiedliche Behandlung von verheirateten
Frauen und Männern die empirisch bereits vielfach festgestellte Heiratsprämie
verursachen könnte.

\subsection{Effekte von Personenmerkmalen auf den “Just Wage Gap”}

In den bisherigen Analysen haben wir jeweils geprüft, ob das Geschlecht der
Versuchsperson einen Einfluss darauf hat, inwieweit das Einkommen von Frauen
und Männern unterschiedliche bewertet wird. In allen drei Experimenten fanden
sich keine Hinweise auf einen solchen Effekt. Dies steht in Einklang mit der
„reward expectations theory“ von \citet{Auspurg-etal-2017}. In
Tabelle~\ref{tab-4} sind die Resultate nochmals zusammengetragen. Zudem gibt
die Tabelle Aufschluss über die Effekte einiger weiterer Personenmerkmale.
Dargestellt ist jeweils der totale Effekt des entsprechenden Personenmerkmals
auf den Unterschied in der Bewertung des Einkommens von Frauen und Männern
(also ohne Kontrolle der anderen Personenmerkmale). Anzumerken ist, dass es
sich hierbei um explorative Analysen handelt, denen kein experimentelles Design
zugrunde liegt (d.h., wenn ein Unterschied gefunden wird, heisst das nicht
zwingen, dass der Unterschied ursächlich auf das entsprechende Personenmerkmal
zurückzuführen ist; der Zusammenhang könnte auch durch eine nicht beobachtete
Drittvariable erzeugt worden sein).


\begin{table}
    \caption{Effekte von Personenmerkmalen auf den “Just Wage Gap”}\label{tab-4}
    \small
    \def\sym#1{\ifmmode^{#1}\else\(^{#1}\)\fi}
    \begin{tabular*}{\textwidth}{@{\extracolsep\fill}l*{3}{D{.}{.}{2.3}c}@{}}
    \toprule
                                  &\multicolumn{2}{c}{Experiment 1}&\multicolumn{2}{c}{Experiment 2}&\multicolumn{2}{c}{Experiment 3}     \\
                                  \cmidrule(lr){2-3}\cmidrule(lr){4-5}\cmidrule(l){6-7}
                                  &\multicolumn{1}{c@{}}{Effekt}&$N$
                                  &\multicolumn{1}{c@{}}{Effekt}&$N$
                                  &\multicolumn{1}{c@{}}{Effekt}&$N$\\
    \expandableinput log/tab4.tex
    \bottomrule
    \end{tabular*}
    \par\medskip\footnotesize 
    Standardfehler in Klammern; \sym{+} \(p<0.1\), \sym{*} \(p<0.05\), \sym{**} \(p<0.01\), \sym{***} \(p<0.001\) (zweiseitig)
\end{table}

Aufgrund der zunehmenden Gleichstellungsbestrebungen im Zuge des
Modernisierungsprozesses über die letzten 50 Jahre könnte man von einem über
die Geburtskohorten vermittelten Wertwandel ausgehen. Dies liesse einen
entsprechenden Effekt des Alters der Versuchspersonen vermuten. Tatsächlich ist
er Alterseffekt in allen drei Experimenten positiv, das heisst, ältere Personen
machen in ihren Bewertungen tendenziell einen etwas grösseren Unterschied
zwischen Frauen und Männern, der Effekt ist jedoch nur schwach ausgeprägt und in
keinem der Experimente signifikant ($p$-Werte von 0.35, 0.84, und 0.12). Eine
weitere Hypothese wäre, dass über Bildung vermittelte Aufklärung zu einer
Reduktion der geschlechtsspezifischen Unterschiede in den Bewertungen führt. In
Übereinstimmung mit dieser Hypothese ist der Bildungseffekt in allen drei
Experimenten negativ, und in zwei Experimenten ist der Effekt auch statistisch
signifikant ($p$-Werte von 0.013, 0.042, und 0.18). Bezüglich des Einkommens
der Versuchsperson (äquivalenzskaliertes Haushaltseinkommen) zeigt sich zwar
ein Ankereffekt (das heisst, Personen mit hohem Einkommen bewerten das
Einkommen in der Vignette eher als zu tief), der “Just Wage Gap” scheint jedoch
nicht mit dem Einkommen zusammenzuhängen. Als letzten Faktor prüfen wir den
Einfluss der politischen Einstellung der Versuchspersonen. Aufgrund der
stärkeren Orientierung an traditionellen Rollenbildern könnte vermutet werden,
dass politisch eher rechts eingestellte Personen einen grösseren
geschlechtsspezifischen Unterschied in den Einkommensbewertungen machen. Der
Effekt kann nur in den ersten beiden Experimenten ermittelt werden, da in der
dritten Studie die politische Orientierung nicht erhoben wurde. In beiden
Experimenten ist der Effekt erwartungsgemäss positiv, jedoch nur im ersten
Experiment statistisch signifikant ($p$-Werte von 0.031 und 0.57).

Insgesamt finden wir in diesen explorativen Analysen nur schwache Hinweise, dass
substanzielle systematische Einflüsse von Personenmerkmalen auf den “Just Wage Gap” bestehen.

\section{Diskussion und Schlussfolgerungen}

In den drei präsentierten Experimenten ging es darum, Mechanismen aufzudecken, die
potenziell für denjenigen Teil der Lohnlücke zwischen Frauen und Männern
verantwortlich sein können, der sich mit herkömmlichen statistischen Analysen
nur schlecht analysieren lässt (meist „unerklärter Anteil“ oder
„Diskriminierungseffekt“ genannt). In Experiment 1 zeigten wir, dass die
Einkommen von identischen Frauen und Männern ungleich bewertet werden. Im
Umkehrschluss bedeutet dies, dass die Befragten tiefere Frauenlöhne als
gerechtfertigt einschätzten. Der Zivilstand aller beschriebenen Personen war
entweder verheiratet oder alleinerziehend. Niemand wurde als ledig und
alleinstehend beschrieben. Wir führten deshalb ein zweites Experiment durch, in
dem wir den Zivilstand konstant hielten und alle Vignetten-Personen als
alleinstehend und ohne Kinder beschrieben. In diesem Experiment konnten wir
keinen Geschlechtereffekt mehr feststellen. Alleinstehende Frauen werden also
gegenüber alleinstehenden Männern nicht benachteiligt. Das im ersten Experiment
beschriebene Bedürftigkeitskriterium enthielt zwei Dimensionen, die familiäre
und die finanzielle Situation. Um dies auseinanderzuhalten führten wir ein
drittes Experiment durch, in dem wir explizit den Einfluss des Zivilstands
testen wollten. Unsere Hypothese diesbezüglich wurde weitgehend bestätigt: Es
gibt eine Heiratsprämie für Männer. Ihnen wird im Gegensatz zu alleinstehenden
Männern und zu Frauen, bei denen der Zivilstand keinen Einfluss hat, ein
höheres Einkommen zugestanden.

Weiter haben wir untersucht, welche Rolle die Prinzipien der
Verteilungsgerechtigkeit in Kombination mit dem askriptiven Merkmal des
Geschlechts bei der Einkommensbewertung spielen. Wie diese Aspekte
zusammenhängen wurde bisher kaum erforscht. Die meisten theoretischen
Diskussionen erwähnen zwar, dass nebst den drei Kriterien Gleichheit, Verdienst
und Bedürftigkeit auch askriptive Merkmale eine Rolle spielen können, sagen
aber weder weshalb, noch wie dies mit den jeweils diskutierten
Gerechtigkeitsprinzipien vereinbar ist (oder eben nicht). Ähnlich verhält es
sich in den empirischen Untersuchungen, die die verschiedenen Kriterien
weitgehend separat testen. Die beiden Konzepte der Verteilungsgerechtigkeit,
Verdienst und Bedürftigkeit, sind aber aus einer gendersensiblen Perspektive
problematisch: Das Verdienst-Kriterium berücksichtigt nicht, dass Frauen und
Männer unterschiedliche Voraussetzungen haben können, die gleiche Leistung zu
erbringen. Der Umstand, dass Frauen neben der Erwerbsarbeit oft auch noch die
Hauptverantwortung für Haushalt und Kinder tragen, hat einen Einfluss auf ihre
Verfügbarkeit auf dem Arbeitsmarkt. Dies zeigt sich vor allem in der unter
Frauen weit verbreiteten Teilzeitarbeit. Hingegen ist davon auszugehen, dass
sich die effektiv geleistete Arbeit in der Qualität nicht von der Arbeit der
Männer unterscheidet. Es kommt also darauf an, woran der Verdienst geknüpft
wird. In Betriebskulturen, in denen der sogenannte „Präsentismus“ hoch
geschätzt wird, haben Frauen, welche um 17 Uhr ihre Kinder von der Krippe
abholen müssen, einen klaren Nachteil im „Wettbewerb“ darum, wer abends am
Arbeitsplatz zuletzt das Licht löscht (vgl. \citealp{Goldin-2014}). Gerade für
den beruflichen Aufstieg sind lange Präsenzzeiten und hohe Flexibilität immer
noch sehr wichtig. Aufgrund der nach wie vor relativ traditionellen
Rollenteilung, bei der die Männer mehr zum Haushaltseinkommen beitragen und die
Frauen den grösseren Teil der Hausarbeit und Kindererziehung erledigen, ist
auch das Bedürftigkeitsprinzip asymmetrisch. Dieser Tatsache wird weder in den
theoretischen Diskussionen noch in ihrer empirischen Anwendung genügend
Rechnung getragen. Mit dem dritten Experiment haben wir in diesem Punkt
versucht einen Beitrag zu leisten und haben gezeigt, wie der Zivilstand, sofern
dieser tatsächlich ein sinnvoller Proxy für Bedürftigkeit ist, für Frauen und
Männer im Zusammenhang mit der Beurteilung eines gerechten Einkommens eine
unterschiedliche Bedeutung hat.


% References

\begin{small}
\bibliographystyle{asr_de}
\bibliography{bib}
\end{small}

% Appendix

\begin{table}
    \caption{Durchschnittliche Einkommensbewertung nach experimentellen Faktoren}\label{tab-A1}
    \small
    \def\sym#1{\ifmmode^{#1}\else\(^{#1}\)\fi}
    \begin{tabular*}{\textwidth}{@{\extracolsep\fill}lD{.}{.}{2.2}D{.}{.}{1.2}D{.}{.}{3.0}D{.}{.}{2.2}D{.}{.}{1.2}D{.}{.}{3.0}D{.}{.}{2.2}D{.}{.}{1.2}@{}}
    \toprule
                                  &\multicolumn{3}{c}{Frauen}            &\multicolumn{3}{c}{Männer}            &\multicolumn{2}{c}{Differenz}     \\
                                  \cmidrule(lr){2-4}\cmidrule(lr){5-7}\cmidrule(l){8-9}
                                  &\multicolumn{1}{c@{}}{$\overline{Y}$}&\multicolumn{1}{c@{}}{$\hat\sigma$}&N
                                  &\multicolumn{1}{c@{}}{$\overline{Y}$}&\multicolumn{1}{c@{}}{$\hat\sigma$}&N
                                  &\multicolumn{1}{c@{}}{$\Delta$}&\multicolumn{1}{c@{}}{$\hat\sigma$}\\
    \expandableinput log/tabA1.tex
    \bottomrule
    \end{tabular*}
    \par\medskip\footnotesize 
    Abhängige Variable: Einkommensbewertung (--5 = „viel zu niedrig“ bis 5 =
    „viel zu hoch“); $\overline{Y}$: Mittelwert;
    $\hat\sigma$: Standardfehler; Differenz $\Delta$: \sym{+} \(p<0.1\),
    \sym{*} \(p<0.05\), \sym{**} \(p<0.01\), \sym{***} \(p<0.001\) (zweiseitig)
\end{table}

\begin{table}
    \caption{Deskriptive Statistiken der Stichproben}\label{tab-A2}
    \small
    \begin{tabular*}{\textwidth}{@{\extracolsep\fill}l*{3}{D{.}{.}{5.2}D{.}{.}{4.3}}@{}}
    \toprule
                                  &\multicolumn{2}{c}{Experiment 1} &\multicolumn{2}{c}{Experiment 2} &\multicolumn{2}{c}{Experiment 3}\\
                                  \cmidrule(lr){2-3}\cmidrule(lr){4-5}\cmidrule(l){6-7}
                                  &\multicolumn{1}{c@{}}{$\overline{X}$}&\multicolumn{1}{c@{}}{$\sigma$}
                                  &\multicolumn{1}{c@{}}{$\overline{X}$}&\multicolumn{1}{c@{}}{$\sigma$}
                                  &\multicolumn{1}{c@{}}{$\overline{X}$}&\multicolumn{1}{c@{}}{$\sigma$}\\
    \expandableinput log/tabA2.tex
    \bottomrule
    \end{tabular*}
    \par\medskip\footnotesize 
    $\overline{X}$: Mittelwert bzw. Prozentanteil; $\sigma$: Standardabweichung;
    Einkommen: Haushaltsäquivalenzeinkommen bei Experiment 1 und 3, persönliches Einkommen bei Experiment 2;
    politische Orientierung: 1 = ganz links, 10 = ganz rechts
\end{table}

\begin{table}
    \caption{Tabelle zu Abbildung \ref{fig-ex1}}\label{tab-ex1}
    \small
    \def\sym#1{\ifmmode^{#1}\else\(^{#1}\)\fi}
    \begin{tabular*}{\textwidth}{@{\extracolsep\fill}lD{.}{.}{2.2}D{.}{.}{1.2}D{.}{.}{2.2}D{.}{.}{1.2}D{.}{.}{2.2}D{.}{.}{1.2}@{}}
    \toprule
                                  &\multicolumn{2}{c}{Frauen}            &\multicolumn{2}{c}{Männer}            &\multicolumn{2}{c}{Differenz}     \\
                                  \cmidrule(lr){2-3}\cmidrule(lr){4-5}\cmidrule(l){6-7}
                                  &\multicolumn{1}{c@{}}{$\widehat E(Y)$}&\multicolumn{1}{c@{}}{$\hat\sigma$}
                                  &\multicolumn{1}{c@{}}{$\widehat E(Y)$}&\multicolumn{1}{c@{}}{$\hat\sigma$}
                                  &\multicolumn{1}{c@{}}{$\Delta$}&\multicolumn{1}{c@{}}{$\hat\sigma$}\\
    \expandableinput log/tab1.tex
    \bottomrule
    \end{tabular*}
    \par\medskip\footnotesize 
    Abhängige Variable: Einkommensbewertung (--5 = „viel zu niedrig“ bis 5 =
    „viel zu hoch“);
    $\widehat E(Y)$: Durchschnittliche Bewertung; $\Delta$: Differenz zwischen Frauen und Männern;
    $\hat\sigma$: Standardfehler\newline
    Differenztests: \sym{+} \(p<0.1\), \sym{*} \(p<0.05\), \sym{**} \(p<0.01\), \sym{***} \(p<0.001\) (zweiseitig)
\end{table}

\begin{table}
    \caption{Tabelle zu Abbildung \ref{fig-ex2}}\label{tab-ex2}
    \small
    \def\sym#1{\ifmmode^{#1}\else\(^{#1}\)\fi}
    \begin{tabular*}{\textwidth}{@{\extracolsep\fill}lD{.}{.}{2.2}D{.}{.}{1.2}D{.}{.}{2.2}D{.}{.}{1.2}D{.}{.}{2.2}D{.}{.}{1.2}@{}}
    \toprule
                                  &\multicolumn{2}{c}{Frauen}            &\multicolumn{2}{c}{Männer}            &\multicolumn{2}{c}{Differenz}     \\
                                  \cmidrule(lr){2-3}\cmidrule(lr){4-5}\cmidrule(l){6-7}
                                  &\multicolumn{1}{c@{}}{$\widehat E(Y)$}&\multicolumn{1}{c@{}}{$\hat\sigma$}
                                  &\multicolumn{1}{c@{}}{$\widehat E(Y)$}&\multicolumn{1}{c@{}}{$\hat\sigma$}
                                  &\multicolumn{1}{c@{}}{$\Delta$}&\multicolumn{1}{c@{}}{$\hat\sigma$}\\
    \expandableinput log/tab2.tex
    \bottomrule
    \end{tabular*}
    \par\medskip\footnotesize 
    Abhängige Variable: Einkommensbewertung (--5 = „viel zu niedrig“ bis 5 =
    „viel zu hoch“);
    $\widehat E(Y)$: Durchschnittliche Bewertung; $\Delta$: Differenz zwischen Frauen und Männern;
    $\hat\sigma$: Standardfehler\newline
    Differenztests: \sym{+} \(p<0.1\), \sym{*} \(p<0.05\), \sym{**} \(p<0.01\), \sym{***} \(p<0.001\) (zweiseitig)
\end{table}

\begin{table}
    \caption{Tabelle zu Abbildung \ref{fig-ex2chf}}\label{tab-ex2chf}
    \small
    \def\sym#1{\ifmmode^{#1}\else\(^{#1}\)\fi}
    \begin{tabular*}{\textwidth}{@{\extracolsep\fill}lD{.}{.}{2.2}D{.}{.}{1.2}D{.}{.}{2.2}D{.}{.}{1.2}D{.}{.}{2.2}D{.}{.}{1.2}@{}}
    \toprule
                                  &\multicolumn{2}{c}{Frauen}            &\multicolumn{2}{c}{Männer}            &\multicolumn{2}{c}{Differenz}     \\
                                  \cmidrule(lr){2-3}\cmidrule(lr){4-5}\cmidrule(l){6-7}
                                  &\multicolumn{1}{c@{}}{$\widehat E(Y)$}&\multicolumn{1}{c@{}}{$\hat\sigma$}
                                  &\multicolumn{1}{c@{}}{$\widehat E(Y)$}&\multicolumn{1}{c@{}}{$\hat\sigma$}
                                  &\multicolumn{1}{c@{}}{$\Delta$}&\multicolumn{1}{c@{}}{$\hat\sigma$}\\
    \expandableinput log/tab2chf.tex
    \bottomrule
    \end{tabular*}
    \par\medskip\footnotesize 
    Abhängige Variable: Einkommensbewertung (--5 = „viel zu niedrig“ bis 5 =
    „viel zu hoch“); Resultate umgerechnet zu CHF;
    $\widehat E(Y)$: Durchschnittliche Bewertung; $\Delta$: Differenz zwischen Frauen und Männern;
    $\hat\sigma$: Standardfehler\newline
    Differenztests: \sym{+} \(p<0.1\), \sym{*} \(p<0.05\), \sym{**} \(p<0.01\), \sym{***} \(p<0.001\) (zweiseitig)
\end{table}

\begin{table}
    \caption{Tabelle zu Abbildung \ref{fig-ex3}}\label{tab-ex3}
    \small
    \def\sym#1{\ifmmode^{#1}\else\(^{#1}\)\fi}
    \begin{tabular*}{\textwidth}{@{\extracolsep\fill}lD{.}{.}{2.2}D{.}{.}{1.2}D{.}{.}{2.2}D{.}{.}{1.2}D{.}{.}{2.2}D{.}{.}{1.2}@{}}
    \toprule
                                  &\multicolumn{2}{c}{Frauen}            &\multicolumn{2}{c}{Männer}            &\multicolumn{2}{c}{Differenz}     \\
                                  \cmidrule(lr){2-3}\cmidrule(lr){4-5}\cmidrule(l){6-7}
                                  &\multicolumn{1}{c@{}}{$\widehat E(Y)$}&\multicolumn{1}{c@{}}{$\hat\sigma$}
                                  &\multicolumn{1}{c@{}}{$\widehat E(Y)$}&\multicolumn{1}{c@{}}{$\hat\sigma$}
                                  &\multicolumn{1}{c@{}}{$\Delta$}&\multicolumn{1}{c@{}}{$\hat\sigma$}\\
    \expandableinput log/tab3.tex
    \bottomrule
    \end{tabular*}
    \par\medskip\noindent\footnotesize 
    Abhängige Variable: Einkommensbewertung (--5 = „viel zu niedrig“ bis 5 =
    „viel zu hoch“);
    $\widehat E(Y)$: Durchschnittliche Bewertung; $\Delta$: Differenz zwischen Frauen und Männern;
    $\hat\sigma$: Standardfehler\newline
    Differenztests: \sym{+} \(p<0.1\), \sym{*} \(p<0.05\), \sym{**} \(p<0.01\), \sym{***} \(p<0.001\) (zweiseitig)
\end{table}

\begin{table}
    \caption{Tabelle zu Abbildung \ref{fig-ex3chf}}\label{tab-ex3chf}
    \small
    \def\sym#1{\ifmmode^{#1}\else\(^{#1}\)\fi}
    \begin{tabular*}{\textwidth}{@{\extracolsep\fill}lD{.}{.}{4.1}D{.}{.}{3.1}D{.}{.}{4.1}D{.}{.}{3.1}D{.}{.}{3.1}D{.}{.}{3.1}@{}}
    \toprule
                                  &\multicolumn{2}{c}{Frauen}            &\multicolumn{2}{c}{Männer}            &\multicolumn{2}{c}{Differenz}     \\
                                  \cmidrule(lr){2-3}\cmidrule(lr){4-5}\cmidrule(l){6-7}
                                  &\multicolumn{1}{c@{}}{$\widehat E(Y)$}&\multicolumn{1}{c@{}}{$\hat\sigma$}
                                  &\multicolumn{1}{c@{}}{$\widehat E(Y)$}&\multicolumn{1}{c@{}}{$\hat\sigma$}
                                  &\multicolumn{1}{c@{}}{$\Delta$}&\multicolumn{1}{c@{}}{$\hat\sigma$}\\
    \expandableinput log/tab3chf.tex
    \bottomrule
    \end{tabular*}
    \par\medskip\footnotesize 
    Abhängige Variable: Einkommensbewertung (--5 = „viel zu niedrig“ bis 5 =
    „viel zu hoch“); Resultate umgerechnet zu CHF;
    $\widehat E(Y)$: Durchschnittliche Bewertung; $\Delta$: Differenz zwischen Frauen und Männern;
    $\hat\sigma$: Standardfehler\newline
    Differenztests: \sym{+} \(p<0.1\), \sym{*} \(p<0.05\), \sym{**} \(p<0.01\), \sym{***} \(p<0.001\) (zweiseitig)
\end{table}





\end{document}


